\documentclass{article}
\usepackage[utf8]{inputenc}
\usepackage[catalan]{babel}% a la uni no funcione

\usepackage{mathtools}
\usepackage{amssymb}

\newcommand{\N}{\mathbb{N}}
\newcommand{\Z}{\mathbb{Z}}
\newcommand{\Q}{\mathbb{Q}}
\newcommand{\R}{\mathbb{R}}
\newcommand{\C}{\mathbb{C}}

\DeclareMathOperator{\Int}{Int}
\DeclareMathOperator{\op}{op}
\DeclareMathOperator{\fl}{fl}

\usepackage{microtype}     % microtypography, reduces hyphenation

\usepackage[font=small,labelformat=simple,]{caption}   % customizing captions

\usepackage{titlesec}      % customizing section titles
\titleformat{\section}{\itshape\large}{}{0em}{}
\titlespacing{\section}{0pt}{8pt}{4pt}
\titleformat{\subsection}{\itshape}{}{0em}{}
\titlespacing{\subsection}{0pt}{4pt}{2pt}
\titleformat{\subsubsection}[runin]{\bf\scshape}{}{0em}{}
\titlespacing{\subsubsection}{0pt}{5pt}{5pt}

\usepackage[papersize={3.6in,4.8in},hmargin=0.1in,vmargin={0.1in,0.1in}]{geometry}  % page geometry

\usepackage{fancyhdr}   % headers and footers
\pagestyle{fancy}
\fancyhead{}            % clear page header
\fancyfoot{}            % clear page footer

\setlength{\abovecaptionskip}{2pt} % space above captions 
\setlength{\belowcaptionskip}{0pt} % space below captions
\setlength{\textfloatsep}{2pt}     % space between last top float or first bottom float and the text
\setlength{\floatsep}{2pt}         % space left between floats
\setlength{\intextsep}{2pt}        % space left on top and bottom of an in-text float
%% he fotut un copy past salvatje



\usepackage{amsmath} % xrightarrow

\title{Mètodes Numerics I\\ Errors}
\author{Sistach Reinsoso, Arnau}

\begin{document}

\maketitle
\tableofcontents

\section{Error}
\subsection{Introducció}
\begin{itemize}
\item $X$ dades
\item $F$ càlculs
\item $Z$ resultat
	\subitem $X \xrightarrow{F} Z$
\end{itemize}

Suposem
\begin{enumerate}
\item $\exists e \in X: \widetilde{X}$
\item $\exists e \in F: \widetilde{F}$
	\begin{itemize}
	\item $\widetilde{Z} = \widetilde{F} (\widetilde{X})$
	\item $\widetilde{Z} - Z =
		\overbrace{\widetilde{F}(\widetilde{X}) -
			F(\widetilde{X})}^F +
		\overbrace{F(\widetilde{X}) - F(X)}^X$
	\end{itemize}
\end{enumerate}


\subsection{Propagació de l'error}
\begin{itemize}
\item Error absolut
	\subitem $\widetilde{X} - X \equiv e, e_a, \Delta$
\item Error relatiu
	\subitem $\frac{e_a}{X} \equiv e_r, \delta \approx \frac{e_a}{\widetilde{X}}$
\end{itemize}
\subsection{Fita}
\begin{itemize}
\item[$\Delta$]: $\forall \varepsilon_a > 0: |\Delta| \le \varepsilon_a$
\item[$e_r$]: $\forall \varepsilon_r > 0: |e_r| \le \varepsilon_r$
\item Notació
	\subitem $x = \tilde{x} \pm \varepsilon_a$ Error absolut
	\subitem $x = \tilde{x} (1 \pm \varepsilon_r)$ Error relatiu
\end{itemize}

\subsection{Arrodonir / tallar}
\begin{itemize}
\item Idea bàsica
	\begin{itemize}
	\item $|e_a| \le \frac{1}{2} b^{-t}$ té $t$ decimals correctes
	\item dígits significatius
	\end{itemize}

\item Enters
	\begin{itemize}
	\item $\Z/(2^n)$
	\item int: $n = 32 \Rightarrow [-2^{-31}, 2^{31} -1] \cap \Z$
	\end{itemize}

\item Reals posicional
	\begin{itemize}
	\item Base $\beta \in \N^+$
		\subitem Dígits $\{0, 1, \dots , \beta -1\}$
		\begin{itemize}
		\item $m_1 \to$ dígits enters
		\item $m_2 \to$ dígits decimals
		\end{itemize}
	\end{itemize}

\item Reals punt flotant
	\begin{itemize}
	\item $\underline{\overline x} \in \R\setminus \{0\}$
	\item $x = M \beta^e$
		\subitem
		$
		\begin{cases}
		M = \{\pm\} D_0, D_1	& D_i \in \{0, 1, \dots, \beta -1\} \\
		e \in \Z		& D_0 \neq 0
		\end{cases}
		$
	\item $e$ finit
	\item $M$ pot ésser infinit $1/3$
	\item float C 32 = 1 signe + 8 exp. + 23 mantisa
	\item $X = M \beta^e \approx x = m \beta^e$
		\subitem $m = \pm d_0.d_1\_d_{t_{(\beta}} \quad d_0 \neq 0$
		\subitem $1 \le |m| < \beta$
	\item Això s'anomena: Sistema de representació de nombres reals $\neq 0$ en punt flotant normalitzat definit per les constants
	\item $(\beta, t, L, U)$
		\begin{itemize}
		\item $\beta$ base
		\item $t$ valors en memòria
		\item $L$ límit inferior
		\item $U$ límit superior
		\end{itemize}
	\item $|m - M| \le \frac{1}{2} \beta^{-t} \Rightarrow |x- X| \le \frac{1}{2} \beta^{-t}$
	\end{itemize}
\item Definició
	\begin{itemize}
	\item $\mu = \frac{1}{2} \beta^{-1}$ unitat d'arrodoniment
	\item $X$ valor representat
	\item $\left|\frac{X-x}{X}\right| \le \mu$
	\end{itemize}

\item Errors d'arrodoniment en les operacions
	\begin{itemize}
	\item $\op \in \{+, -, \times, /\}$
	\item $\widetilde{x\op y} = \fl(x\op y)$
	\item Suposem el càlcul exacte, però el enmagatzemar no
	\item Axioma, general, suposem cert
	\item $\left|\frac{\fl(x\op y) - (x\op y)}{x\op y}\right| \le \mu$
	\end{itemize}

\item Propietats falses
	\begin{itemize}
	\item[Sí] Commutativa
	\item[No] Associativa $(a+b)+c \neq a+(b+c)$
	\item[No] Distributiva
	\end{itemize}

\item Per les funcions
	\begin{itemize}
	\item Sigui $f \in \{ \cos, \sin, \exp, \log \dots \}$
	\item $\left|\frac{\fl(f(x)) - f(x)}{f(x)}\right| \le k\mu$
		\subitem Profe se queixe que no ha vist desde fa molt la $k$ enlloc
	\end{itemize}
\item Si hi ha moltes operacions ``difícil''
	\begin{itemize}
	\item Suposem cert l'Axioma $\fl(a+b) = (a+b)(1+\varepsilon)$ i $|\varepsilon| \le \mu$
	$$\widetilde{S_n} = x_1 \prod_{k=2}^n (1 + \varepsilon_k) + \sum_{i = 2}^n x_i\left(\prod_{k = i}^n (1+ \varepsilon_k)\right)$$
	\item Notació $\eta_1, \dots, \eta_n$
	$$\widetilde{S_n} = \sum_{i=2}^n x_i(1 + \eta_i)$$
	\end{itemize}

\item Anàlisi de l'error endarrera
	\begin{itemize}
	\item La solució aproximada $\widetilde{S_n}$ obtinguda amb $\op$ aproximades és pensada com la solució exacta amb dades aproximades
	\item[$i = 1$] $1+\eta_1 = \prod_{k=2}^n (1 + \varepsilon_k)$
	\item[$i \neq 1$] $1+\eta_i = \prod_{k=i}^n (1 + \varepsilon_k)$
	\item Suposant $|\varepsilon_i| \le \mu \ll 1 \quad \varepsilon_i \varepsilon_j \approx 0$
	\item O sigui: $|\eta_i| \lesssim (n -i +1)\mu \quad i \neq 1$
	\item $|\eta_1| \lesssim (n -1)\mu$
	\end{itemize}
\item Interpretació
	\begin{itemize}
	\item Quan se suma molts terrmes positius, cal sumar de petit a gran
	\end{itemize}
\end{itemize}

\subsection{Propagació de l'error en variables}
\begin{itemize}
\item 1 variable
	\begin{itemize}
	\item Teorema
		$$
		\begin{rcases}
			f:I \subset \R \to \R \text{derivable}\\
			x, \tilde{x} \in I, x \neq \tilde{x}
		\end{rcases}
		\Rightarrow \exists \xi \in \Int\!\!<\!\!X, \widetilde{X}\!\!>
		|$$$$
		f(\widetilde{X}) - f(X) = f'(\xi) (\widetilde{X} - X)
 		$$
	\item Corol·lari
		\begin{enumerate}
		\item $\forall x, \widetilde{x} \in I$
			$$|f(\tilde{x}) - f(x)| \le \max_{\xi \in \Int\!\!<\!\!x,\tilde{x}\!\!>}|f'(\xi)||\Delta|$$
		\item aproximació de 1er ordre Taylor
			$$|\Delta f| \approx |f'(\tilde{x})||\Delta x|$$
		\end{enumerate}
	\end{itemize}
\item n variables
	\begin{itemize}
	\item Teorema
		$$
		\begin{rcases}
			f:I_1 \times I_2 \times \dots \times I_n \subset \R^n \to \R\\
			x, \tilde{x} \in I\\
			\Delta x = (\Delta x_1, \dots, \Delta x_n)
		\end{rcases} \Rightarrow
		$$
		$$
		\exists \theta \in (0, 1) | f(\tilde{x}) - f(x) =
		\sum^n_k \frac{\delta f}{\delta x_k} (x + \theta \Delta x)\Delta x_k
		$$
	\item Corrol·lari
		\begin{enumerate}
		\item $|\Delta f| \le \sum_k \max |\frac{\delta f}{\delta x_k}||\Delta x_k|$
		\item $\Delta f \approx \sum \frac{\delta f}{\delta x_k}(\tilde{x})\Delta x_k$
			\subitem Però llavors
			\subitem $|\Delta f| \lesssim \sum \left|\frac{\delta f}{\delta x_k}(\tilde{x})\right||\Delta x_k|$
		\end{enumerate}
	\end{itemize}
\end{itemize}

\subsection{Cance\lgem ació}
Restar dos valors proxims.
\begin{itemize}
\item $y \approx 0$
\item $\frac{\Delta y}{y} = \frac{\Delta x_1 - \Delta x_2}{y}$
\item Error absolut és igual
\item Error relatiu incrementa molt
\end{itemize}

\subsection{Exemples}
Operacions aritmètiques
\begin{itemize}
\item Suma
	\begin{itemize}
	\item
		$
		\begin{rcases}
		y = x_1 + x_2\\
		\tilde y = \tilde x_1 + \tilde x_2
		\end{rcases}
		\tilde y - y = \tilde x_1 + \tilde x_2 - (x_1 + x_2)
		$
	\item $\Delta y = \Delta x_1 + \Delta x_2$
	\item $|\Delta y| = |\Delta x_1 + \Delta x_2| \le |\Delta x_1| + |\Delta x_2| \le \varepsilon_1 + \varepsilon_2$
	\end{itemize}

\item Producte
	\begin{itemize}
	\item
		$
		\begin{rcases}
		y = x_1 \cdot x_2\\
		\tilde y = \tilde x_1 \cdot \tilde x_2
		\end{rcases}
		$
	\item $\tilde y - y = (x_1 + \Delta x_1)(x_2 + \Delta x_2) - x_1 x_2$
	\item $\Delta y \approx x_1 \Delta x_2 + x_2 \Delta x_1$
	\item Error relatiu
		\subitem $\frac{\Delta y}{y} \approx \frac{\Delta x_1}{x_1} + \frac{\Delta x_2}{x_2}$
		\subitem $|\frac{\Delta y}{y}| \lesssim |\frac{\Delta x_1}{x_1}| + |\frac{\Delta x_2}{x_2}|$
	\end{itemize}

\item Divisió
	\begin{itemize}
	\item like
	\end{itemize}
\end{itemize}

\subsection{$\sin(x^2_1 \cdot x_2)$}
\begin{itemize}
	\item $x_1 = 0.75 \pm 10^{-2}$
	\item $x_2 = 0.413\pm 10^{-3}$
	\item $y = f(x_1 \cdot x_2) = \sin (x_1^2 x_2)$
	\item $\frac{\delta f}{\delta x_1} = (2 x_1 x_2)\cos(x_1^2x_2) \approx 0.6029$
	\item $\frac{\delta f}{\delta x_2} = x_1^2 \cos(x_1^2x_2) \approx 0.5474$
		\subitem $|\tilde y - y| \lesssim |\frac{\delta f}{\delta x_1} \varepsilon_1| + |\frac{\delta f}{\delta x_2} \varepsilon_2|
			\le 0.8 \cdot 10^{-2}$
\end{itemize}
Si vols ser més exacte, pots veure si és monotona. Així tens una presició més bona. Tot i que dona generalment més problemes de càlcul.

\subsection{Relacions de recurrència}
Horner
\begin{itemize}
\item cost
	\subitem n productes
	\subitem n sumes
\item Algoritme
\item $p = a_n$
\item $aux = 1$
\item $\forall i = n-1, n-1, \dots , 0$
	\begin{itemize}
	\item $aux *= x_1$
	\item $p += a_i * aux$
	\end{itemize}
\item cost
	\subitem $n$ sumes
	\subitem $2n$ productes

\end{itemize}

\subsubsection{Exemple}
Volem calcular
$$I_n = \int_0^1 \frac{x^n}{x+5} dx\quad n = 1, 2, \dots, 7$$
Deduïm una recurrència
$$I_n = \int_0^1 \frac{(x^n + 5x^{n-1}) - 5x^{n-1}}{x + 5}dx =$$
$$\int_0^1 x^{n-1}dx -5\int_0^1 \frac{x^{n-1}}{x+5}dx = \frac{1}{n} - 5I_{n-1}$$
$$I_n = \frac{1}{n} -5I_{n-1}$$
$I_0 = \left. \ln (x+5)\right]_0^1 = \ln 6 -\ln 5 = \ln\frac{6}{5} \approx 0.18232$\\
Si ara calculem $I_7 = -0.04$, negatiu, cosa que no hauria.\\
Si volem evitar que els errors decrementin:
$$I_{n-1} = \frac{1/n - I_n}{5}$$
Ara hem de fer una aproximació, ja que fer la integral és complex
$$\frac{1}{6} \frac{1}{n+1} = \int_0^1 \frac{x^n}{6} dx \le I_n \le \int_0^1 \frac{x^n}{5}dx = \frac{1}{5} \frac{1}{n+1}$$
$$I_{14} \in \left[\frac{1}{90}, \frac{1}{75}\right]\quad \widetilde{I}_{14} = \frac{11}{900} \approx 0.01222$$
$$|e_a| \le \frac{1}{2}\left[ \frac{1}{75} - \frac{1}{90}\right] = \frac{1}{900} \approx 1.1\times 10^{-2}$$
Un cop fet de $I_{14}$ fins a $I_7$:
$$|error| \le 1.1\times 10^{-2} (5^{-1})^7$$


\end{document}
